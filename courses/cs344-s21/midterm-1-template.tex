\documentclass{article}

\setlength{\headsep}{0.75 in}
\setlength{\parindent}{0 in}
\setlength{\parskip}{0.1 in}

%=====================================================
% Add PACKAGES Here (You typically would not need to):
%=====================================================

\usepackage[margin=1in]{geometry}
\usepackage{amsmath,amsthm}
\usepackage{fancyhdr}
\usepackage{enumitem}
\usepackage{graphicx}
%=====================================================
% Ignore This Part (But Do NOT Delete It:)
%=====================================================

\theoremstyle{definition}
\newtheorem{problem}{Problem}
\newtheorem*{fun}{Fun with Algorithms}
\newtheorem*{challenge}{Challenge Yourself}
\def\fline{\rule{0.75\linewidth}{0.5pt}}
\newcommand{\finishline}{\vspace{-15pt}\begin{center}\fline\end{center}}
\newtheorem*{solution*}{Solution}
\newenvironment{solution}{\begin{solution*}}{{} \end{solution*}}
\newcommand{\grade}[1]{\hfill{\textbf{($\mathbf{#1}$ points)}}}
\newcommand{\thisdate}{\today}
\newcommand{\thissemester}{\textbf{Rutgers: Spring 2021}}
\newcommand{\thiscourse}{CS 344: Design and Analysis of Computer Algorithms} 
\newcommand{\thishomework}{Number} 
\newcommand{\thisname}{Name} 
\newcommand{\thisextension}{Yes/No} 

\headheight 40pt              
\headsep 20pt
\renewcommand{\headrulewidth}{0pt}
\lhead{\small \textbf{Only for the personal use of students registered in CS 344, Spring 2021 at Rutgers University. Redistribution out of this class is strictly prohibited.}}
\pagestyle{fancy}

\newcommand{\thisheading}{
   \noindent
   \begin{center}
   \framebox{
      \vbox{\vspace{2mm}
    \hbox to 6.28in { \textbf{\thiscourse \hfill \thissemester} }
       \vspace{8mm}
       \hbox to 6.28in { {\Large \hfill  Midterm Exam \#\thishomework \hfill} }
       \vspace{2mm}
         \hbox to 6.28in { {\hfill Due: Tuesday, March 2nd, 9:00am EST \hfill} }
       \vspace{4mm}
       \hbox to 6.28in { \emph{Name: \underline{~~~~~~~~~~~~~~~~~~~~~~~~~~~~~~~~~~~~~~~~~~~~} \thisname \hfill NetID: \underline{~~~~~~~~~~~~~~~~~~~~} \thisextension}}
      \vspace{2mm}}
      }
   \end{center}
   \bigskip
}

%=====================================================
% Some useful MACROS (you can define your own in the same exact way also)
%=====================================================


\newcommand{\ceil}[1]{{\left\lceil{#1}\right\rceil}}
\newcommand{\floor}[1]{{\left\lfloor{#1}\right\rfloor}}
\newcommand{\prob}[1]{\Pr\paren{#1}}
\newcommand{\expect}[1]{\Exp\bracket{#1}}
\newcommand{\var}[1]{\textnormal{Var}\bracket{#1}}
\newcommand{\set}[1]{\ensuremath{\left\{ #1 \right\}}}
\newcommand{\poly}{\mbox{\rm poly}}

\newcommand{\leasttwoalg}{\textnormal{\texttt{FIND-SMALLEST-TWO}}}
\newcommand{\totalsum}{\textnormal{\texttt{TOTAL-SUM}}}
\newcommand{\maxalg}{\textnormal{\texttt{MAX-ALG}}}
\newcommand{\minalg}{\textnormal{\texttt{MIN-RAND-ALG}}}

%=====================================================
% Fill Out This Part With Your Own Information:
%=====================================================


\renewcommand{\thishomework}{1} %Homework number
\renewcommand{\thisname}{} % Your name
\renewcommand{\thisextension}{} % Pick only one of the two options accordingly

\begin{document}

\thisheading

\vspace{-0.5cm}
\subsection*{Instructions}

\begin{enumerate}
	\item Do not forget to write your name and NetID above, and to sign Rutgers honor pledge below. 
	\item The exam contains $5$ problems worth $100$ points in total \emph{plus} one {extra} credit problem worth $10$ points. 
	\item This is a take-home exam. You have until Tuesday, March 2nd, 9:00am EST to finish the exam. 
	\item The exam should be done \textbf{individually} and you are not allowed to discuss these questions with anyone. This includes asking any questions or clarifications
	regarding the exam from other students or posting them publicly on Piazza (any inquiry should be sent directly to the Instructor or posted privately on Piazza). You may however consult all
	the materials used in this course (video lectures, notes, textbook, etc.) while writing your solution, but \textbf{no other resources are allowed}.

	\item Remember that you can leave a problem (or parts of it) entirely blank and receive $25\%$ of the grade for that problem (or part). However, this should not  discourage you from attempting a problem if you think 
	you know how to approach it as you will receive partial credit more than $25\%$ as long as you are on the right track. But keep in mind that if you simply do not know the answer, writing a totally wrong answer may lead to $0\%$ credit.
	
	The only \textbf{exception} to this rule is the extra credit problem: you do {not get any credit for leaving the extra credit problem blank}, and it is harder to get partial credit on that problem.
	
	\item \textbf{You should always prove the correctness of your algorithm and analyze its runtime.} Also, as a general rule, avoid using complicated pseudo-code and instead explain your algorithm in English. 
	\item You may use any algorithm presented in the class or homeworks as a building block for your solutions. 
\end{enumerate}

\finishline

\paragraph{Rutgers honor pledge:} Please sign the Rutgers honor pledge below. 

\begin{quote}
\emph{On my honor, I have neither received nor given any unauthorized assistance on this
examination.} 
\end{quote}
\hfill{Signature:\underline{~~~~~~~~~~~~~~~~~~~~~~~~~~~~~~~~~~~~~~~~~~~~~}}

\bigskip

\begin{center}
\begin{tabular}{|c|r|c|}
\hline
Problem. \# & Points & Score \\ \hline\hline
$1$ & 20 & ~~~~~~~~~~~\\  \hline
$2$ & 20 & \\ \hline
$3$ & 20 & \\ \hline
$4$ & 20 & \\ \hline
$5$ & 20 & \\ \hline
$6$ & +10 & \\ \hline
Total & $100 + 10$ & \\ \hline
\end{tabular}
\end{center}

\newpage

\begin{problem}\label{basics}~
\begin{enumerate}[label=(\alph*)]
	\item Determine the \emph{strongest} asymptotic relation between the functions 
	\[
	f(n) = \sqrt{\log{n}} \quad \text{and} \quad g(n) = \frac{n}{2^{(\log\log{n})}},
	\]
	i.e., whether $f(n) = o(g(n))$, $f(n) = O(g(n))$, $f(n) = \Omega(g(n))$, $f(n) = \omega(g(n))$, or $f(n) = \Theta(g(n))$. 
	Remember to prove the correctness of your choice. \grade{10}

\bigskip
\begin{solution}
	Solution to part (a) goes here. 
\end{solution}

	\newpage
	\item Use the \emph{recursion tree} method to solve the following recurrence $T(n)$ by finding the \emph{tightest} function $f(n)$ such that $T(n) = O(f(n))$:   \grade{10}
	\begin{align*}
		T(n) &\leq 8 \cdot T(n/2) + O(n^3). 
	\end{align*} 
	(You do \emph{not} have to prove that your function is the tightest one.) 
	\bigskip
	\begin{solution}
	Solution to part (b) goes here. 
	%--------
	% If you place a "tree-1-b.pdf" file containing the figure to your recursion tree, it will be shown here. If you like to insert your tree any other way, or do not like to draw one at all, please delete all this line of code below. 
	%--------
	\begin{figure}[h!]
			\centering
			\IfFileExists{tree-1-b.pdf}{\includegraphics[width=0.5\textwidth]{tree-1-b.pdf}}{No Figure Yet}
		\caption{Recursion tree for the recurrence in Problem 1-(b).} 
	\end{figure}
\end{solution}

\end{enumerate}
\end{problem}

\newpage


\begin{problem}\label{induction}
	Consider the algorithm below for finding the total sum of the numbers in any array $A[1:n]$.  
	
	\medskip
	
	$\totalsum (A[1:n])$: \vspace{-0.4cm}
	\medskip
	\begin{enumerate}
		\item If $n=0$: return $0$. 
		\item If $n=1$: return $A[1]$.
		\item Otherwise, let $m_1 \leftarrow \totalsum(A[1:\frac n2])$ and $m_2 \leftarrow \totalsum(A[\frac n2+1:n])$. 
		\item Return  $m_1 + m_2$. 
	\end{enumerate}
	We analyze \totalsum~ in this question. 
	\begin{enumerate}[label=(\alph*)]
		\item Use \emph{induction} to prove the correctness of this algorithm. \grade{10}

\bigskip
		\begin{solution}
	Solution to part (a) goes here. 
	\end{solution}


		
		\newpage
		\item Write a recurrence for this algorithm and solve it to obtain a tight upper bound on the worst case runtime of this algorithm. You can use any method you like for solving this recurrence. 		\grade{10}
	\bigskip	
				\begin{solution}
	Solution to part (b) goes here. 
	%--------
	% If you place a "tree-2-b.pdf" file containing the figure to your recursion tree, it will be shown here. If you like to insert your tree any other way, or do not like to draw one at all, please delete all this line of code below. 
	%--------
	\begin{figure}[h!]
			\centering
			\IfFileExists{tree-2-b.pdf}{\includegraphics[width=0.5\textwidth]{tree-2-b.pdf}}{No Figure Yet}
		\caption{Recursion tree for the algorithm in Problem 2-(b).} 
	\end{figure}
\end{solution}


	\end{enumerate}
\end{problem} 

\newpage

\begin{problem}\label{sort}
	You are given a collection of $n$ integers $a_1,\ldots,a_n$ with positive weights $w_1,\ldots,w_n$. For any number $a_i$, we define the \emph{bias} of $a_i$ as:
	\[
		bias(a_i) = |\hspace{-0.1cm}{\sum_{j: a_j < a_i} w_j} - \sum_{k: a_k \geq a_i} w_k|;
	\]
	i.e., the absolute value of the difference between the weights of elements smaller than $a_i$ and the remaining ones. 
	Design and analyze an algorithm that in $O(n\log{n})$ time 
	finds the element that has the \emph{smallest} bias. You can assume that the input is given in two arrays $A[1:n]$ and $W[1:n]$ where $a_i = A[i]$ and $w_i = W[i]$.  

	
	\paragraph{Examples:} 
	\begin{itemize}
	\item When $n=5$, and $A = [1,5,3,2,7]$ and $W=[3,6,2,8,9]$, the smallest biased element is $a_2 = A[2] = 5$ with $bias(a_2) = |(3+2+8)-(6+9)| = 2$.   
	\item When $n=5$, and $A = [1,2,3,4,5]$ and $W=[8,6,5,2,6]$, the smallest biased element is $a_3 = A[3] = 3$ with $bias(a_3) = |(8+6)-(5+2+6)| = 1$. 
	\end{itemize}
	
	\begin{enumerate}[label=(\alph*)]
		\item \emph{Algorithm:} \grade{7}
	
	\bigskip	
	\begin{solution}
	Solution to part (a) goes here. 
	\end{solution}
	
		\newpage
		\item \emph{Proof of Correctness:} \grade{10}
		
		\bigskip	
	\begin{solution}
	Solution to part (b) goes here. 
	\end{solution}
	
	
		\item \emph{Runtime Analysis:} \grade{3}
		
		\bigskip	
	\begin{solution}
	Solution to part (c) goes here. 
	\end{solution}
	
		
	\end{enumerate}
\end{problem}
\newpage

\begin{problem}\label{hash}
	You are given three arrays $A[1:n]$, $B[1:n]$, and $C[1:n]$ of positive integers. The goal is to decide whether or not there are indices $i,j,k \in [1:n]$ such that $A[i] \cdot B[j] = C[k]$; in other words, is it the case that there are numbers in $A$ and $B$ whose multiplication belongs to $C$. 
	
	
	\paragraph{Examples:} 
	\begin{itemize}
		\item When $n=3$, and $A = [1,3,4]$, $B=[2,3,5]$, and $C=[1,3,5]$, the answer is \emph{Yes}, because for instance we have $A[1] \cdot B[3] = C[3]$ or $A[1] \cdot B[2] = C[2]$. 
		\item When $n=3$ and $A = [1,3,4]$, $B=[2,4,6]$, and $C=[7,9,11]$, the answer is \emph{No}. 
	\end{itemize} 
	
	\begin{enumerate}[label=$(\alph*)$]
	\item Suppose all the numbers in $C$ belong to the set $\set{1,2,\ldots, n^2}$. Design and analyze an algorithm with \textbf{worst-case runtime} of $O(n^2)$ for the problem in this case. \grade{10} 
	\bigskip	
	\begin{solution}
	Solution to part (a) goes here. 
	\end{solution}
	
	
	\newpage
	\item Now suppose $C$ can be any arbitrary array of $n$ integers. Design and analyze a \textbf{randomized} algorithm with \textbf{expected worst-case runtime} of $O(n^2)$ for the problem in this case. \grade{10} 

	\medskip
	\emph{Note:} Actually, this problem also has a deterministic algorithm that runs in worst-case $O(n^2)$ time. But you do not need to design such an algorithm for this problem (although if you do, you will receive the full credit for both parts $(a)$ and $(b)$).
	
	\bigskip	
	\begin{solution}
	Solution to part (b) goes here. 
	\end{solution}
	
	
	\end{enumerate}
	
	
\end{problem}

\newpage
\begin{problem}\label{DP-greedy}
Please solve the problems in \textbf{exactly one} of the two parts below. 

\paragraph{Part 1 (dynamic programming):} We want to purchase an item of price $M$ and for that we have a collection of $n$ different coins in an array $C[1:n]$ where coin $i$ has value $C[i]$ (we only have one copy of each coin).  
	Our goal is to purchase this item using the \emph{smallest} possible number of coins or outputting that the item cannot be purchased with these coins. 
	Design a \textbf{dynamic programming} algorithm for this problem with worst-case runtime of $O(n \cdot M)$. \grade{20}
	
	\paragraph{Examples:}
	\begin{itemize}
		\item Given $M=15$, $n=7$, and $C = [4,9,3,2,7,5,6]$, the correct answer is $2$ by picking $C[2]=9$ and $C[7]=6$ which add up to $15$. 
		\item Given $M=11$, $n=4$, and $C = [4,3,5,9]$, the correct answer is that `the item cannot be purchased' as no combination of these coins adds up to a value of $11$ (recall that we can only use each coin once). 
	\end{itemize}
	
	\begin{enumerate}[label=(\alph*)]
	\item \emph{Specification of recursive formula for the problem (in plain English)}: \grade{5}
	
	\bigskip	
	\begin{solution}
	Solution to part (a) goes here. 
	\end{solution}
	
	
	\item \emph{Recursive solution for the formula and its proof of correctness:} \grade{7}
	
	\bigskip	
	\begin{solution}
	Solution to part (b) goes here. 
	\end{solution}
	
	
	\newpage
	\item \emph{Algorithm (either memoization or bottom-up dynamic programming):} \grade{3}
	
	\bigskip	
	\begin{solution}
	Solution to part (c) goes here. 
	\end{solution}
	
	\item \emph{Runtime Analysis:} \grade{5}
	
	\bigskip	
	\begin{solution}
	Solution to part (d) goes here. 
	\end{solution}
	
	
\end{enumerate}


\newpage
\paragraph{Part 2 (greedy):} 
	We want to purchase an item of price $M$ and for that we have an unlimited (!) supply of $\ceil{\log{M}}$ types of coins with value $1,2,4,\cdots,2^{i},\cdots,2^{\ceil{\log{M}}}$. 
	Our goal is to purchase this item using the \emph{smallest} possible number of coins (it is always possible to buy this item by picking $M$ coins of value $1$).
	Design and analyze a \textbf{greedy} algorithm for this problem with $O(\log{M})$ runtime. \grade{20}

	\paragraph{Examples:} 
	\begin{itemize}
		\item Given $M=15$ (and so $\ceil{\log{M}} = 4$), the correct answer is $4$ coins by picking one copy of each of the coins $8,4,2,1$. Note that here we cannot pick the coin of value $2^{\ceil{\log{M}}} = 2^{4} = 16$.
		\item Given $M=32$ (and so $\ceil{\log{M}} = 5$), the correct answer is $1$ coin by picking one copy of the coin $32 = 2^{5} = 2^{\ceil{\log{M}}}$. 

	\end{itemize}
		
		\begin{enumerate}[label=(\alph*)]
	\item \emph{Algorithm}: \grade{5}
	
	\bigskip
				\begin{solution}
	Solution to part (a) goes here. 
	\end{solution}
	
	
	\newpage
	\item \emph{Proof of Correctness}: \grade{10}
	\bigskip
				\begin{solution}
	Solution to part (b) goes here. 
	\end{solution}
	
	\bigskip
	
	\item \emph{Runtime Analysis:} \grade{5}
	\bigskip
				\begin{solution}
	Solution to part (c) goes here. 
	\end{solution}
	
	
\end{enumerate}

\end{problem}
\newpage

\begin{problem}\label{extra}[\textbf{Extra credit}]
	You are given two \emph{unsorted} arrays $A[1:n]$ and $B[1:n]$ consisting of $2n$ \emph{distinct} numbers such that $A[1] < B[1]$ but $A[n] > B[n]$. Design and analyze an algorithm that in $O(\log{n})$ time 
	finds an index $j \in [1:n]$ such that $A[j] < B[j]$ but $A[j+1] > B[j+1]$. 
	
	\emph{Hint:} Start by convincing yourself that such an index $j$ always exist in the first place. \grade{+10}

\end{problem}

\bigskip

			\begin{solution}
	Solution to Problem 6 goes here. 
	\end{solution}
	
\newpage
\subsection*{Extra Workspace}


\end{document}





